\documentclass[a4paper]{article}
\usepackage[utf8]{inputenc}
\usepackage[fleqn]{amsmath}
\usepackage{amssymb}
\usepackage{mathtools}
\usepackage{amsfonts}
\usepackage{lastpage}
\usepackage{tikz}
\usepackage{float}
\usepackage{textcomp}
\usetikzlibrary{patterns}
\usepackage{pdfpages}
\usepackage{gauss}
\usepackage{fancyvrb}
\usepackage[table]{colortbl}
\usepackage{fancyhdr}
\usepackage{graphicx}
\usepackage[margin=2.5 cm]{geometry}

\delimitershortfall-1sp
\newcommand\abs[1]{\left|#1\right|}

\definecolor{listinggray}{gray}{0.9}
\usepackage{listings}
\lstset{
	language=,
	literate=
		{æ}{{\ae}}1
		{ø}{{\o}}1
		{å}{{\aa}}1
		{Æ}{{\AE}}1
		{Ø}{{\O}}1
		{Å}{{\AA}}1,
	backgroundcolor=\color{listinggray},
	tabsize=2,
	rulecolor=,
	basicstyle=\scriptsize,
	upquote=true,
	aboveskip={0.2\baselineskip},
	columns=fixed,
	showstringspaces=false,
	extendedchars=true,
	breaklines=true,
	prebreak =\raisebox{0ex}[0ex][0ex]{\ensuremath{\hookleftarrow}},
	frame=single,
	showtabs=false,
	showspaces=false,
	showlines=true,
	showstringspaces=false,
	identifierstyle=\ttfamily,
	keywordstyle=\color[rgb]{0,0,1},
	commentstyle=\color[rgb]{0.133,0.545,0.133},
	stringstyle=\color[rgb]{0.627,0.126,0.941},
  moredelim=**[is][\color{blue}]{@}{@},
}
\newcommand{\comment}[1]{%
  \text{\phantom{(#1)}} \tag{#1}}
\lstdefinestyle{base}{
  emptylines=1,
  breaklines=true,
  basicstyle=\ttfamily\color{black},
}

\pagestyle{fancy}
\def\checkmark{\tikz\fill[scale=0.4](0,.35) -- (.25,0) -- (1,.7) -- (.25,.15) -- cycle;}
\newcommand*\circled[1]{\tikz[baseline=(char.base)]{
            \node[shape=circle,draw,inner sep=2pt] (char) {#1};}}
\newcommand*\squared[1]{%
  \tikz[baseline=(R.base)]\node[draw,rectangle,inner sep=0.5pt](R) {#1};\!}
\cfoot{Page \thepage\ of \pageref{LastPage}}
\DeclareGraphicsExtensions{.pdf,.png,.jpg}
\author{Ola Rønning (vdl761) \\ Tobias Hallundbæk Petersen (xtv657)}
\title{Advanced Computer Systems \\ Assignment 1}
\lhead{Advanced Computer Systems}
\rhead{Assignment 1}

\begin{document}
\maketitle
\section{Fundamental Abstractions}
\begin{enumerate}
  \item Let us assume that at initialization the administrator of the system knows, how much the system might be scaled up, let us call this $S$, thus the total amount of machines ever scaled to will be at most $SK$. We will then have an array of size $SK$, this will hold communication info necessary for read and write to the individual machines. This array along with its information will reside in the api of the user, giving them direct access to the individual machines. Upon scaling up the new machine will broadcast out to all other machines, which will then update the data-structure at the user upon the next read or write request.\\
    In the case that a machine goes down, the given request will time out and the user will be notified of the error. As there is no request for redundancy in the abstraction, this way of handling it is acceptable.
    The only centralized component in the design is the data-structure at the individual user. And as explained before, any changes done in the form of adding or removing machines will be relayed back to the user in the next request.
  \item For the API, we will denote $n = \lceil \lg (SK) \rceil$, \texttt{\&} will denote bitwise AND, \texttt{>>} denotes bit-shift and lastly all addresses and $n$ is considered as unsigned integers.\\
    The read will address find the machines by masking the $n$ first bits of the given address, the object at this array will expose a function \texttt{requestRead(int)} that will send the request and \texttt{response} which will loop until a response is given or timeout is reached. The response can hold a function that needs to be applied to the current array of machines.
    \begin{lstlisting}[language=Ruby, caption={READ}]
read(addr)
  machine = machines[addr & (n^2 - 1)]
  machine.requestRead(addr >> n)
  case machine.response of
    value:
      return value
    (changes, value):
      machines = changes(machines)
      return value
    time out:
      raise machineTimeoutException
    \end{lstlisting}
    The write is more or less the same as the read, only differing in the added functionality that the user supplies the value to be written.
    \begin{lstlisting}[language=Ruby, caption={WRITE}]
write(addr, value)
  machine = machines[addr & (n^2 - 1)]
  machine.requestWrite(addr >> n, value)
  case machine.response of
    ok:
      return
    (changes):
      machines = changes(machines)
      return
    time out:
      raise machineTimeoutException
    \end{lstlisting}
  \item
    READ/WRITE operations against regular main memory is atomic, the operations against our memory abstraction is also atomic. We have achieved that by always waiting on a response from the machines, whether or not the operation went through, or if it failed, meaning it timed out.
  \item
    In the case that a machine goes down, the data at the given machine would be missing, thus no mapping strategy without redundancy will not be able to retrieve lost data. The memory locations will thus be unavailable, which is what a user could expect from our system, getting an error as opposed to getting the wrong answer is preferable. A new server added that might fill that gap created by a crashing machine, would then be broadcast out in the next request, and the user would be able to address the space again.
\end{enumerate}
\section{Techniques for Performance}
\begin{enumerate}

\item Utilizing concurrency will not influence the latency of unit task a computer system performs, where a unit task is task that has no dependencies on other task the system performs. However, for system tasks that perform several concurrent unit tasks as subtask will have its latency reduced by as some of these will be performed simultaneously. The latency reduction only holds when the overhead of performing unit tasks concurrently is smaller than the gain from concurrent execution.
\item \paragraph{Dallying} is when a system tasks wait with performing a subtask either to batch the request together with later requests or to completely avoid performing the request. An example of dallying is lazy evalutation of programs, where evalutation of expression are deffered until their needed by other computations.
\paragraph{Batching} is when a system task performs a subtask on a group on of requests intead of a single request. Batching is often deployed as an optimization for expensive subtasks such as communicating over a network or lookups in an enormous table. An example of batching is when acknowledgements are piggybacked in TCP.
\item Chaching is an example of fast-path optimization. Cahching makes retriaval time of previously processed request faster by storing their results, thereby creating a fast path for cached results.
\end{enumerate}
\section{Discussion for Architecture}
\subsection{Implementation and testing description}
\paragraph{Testing} is performed on the implemented business logic both as local calls (LC) and remote procedure calls (RPC), only by changing the setup of the testing class. This is reflected in code where the implementation of the unite test are in the abstract class, \texttt{BookStoreTest}, which is extended by the \texttt{HTTPTest} class and the \texttt{LocalTest} class which perform the before mentioned setup to handle LC or RPC respectively. For RPC the implementation spawns a new Java thread that runs the server before the class is constructed, and then sends a stop to the server as part of the tear down of the class. All the implemented tests check that the state of the book-store after a mutation is performed is consistent with how the book-store looked before the mutation.
We have implemented the following test to demonstrate correctness of the implemented \textit{rateBooks} method:
\begin{itemize}
\item \textit{testRateNoneExistingISBN} test that only books that are in the store can be rated. Rating a book not present in the book-store throws an error message and the after state should therefore be the same as the before state, as none of the rating should be committed.
\item \textit{testRateNonsenseISBN} test that only valid ISBN can be used to rate books, in the implementation of \texttt{rateBooks} this check is performed to before checking whether the book is present in the book-store. For the same reasoning as with \texttt{testRateNoneExistingISBN} testing method the state of the bookstore should not change after attempting to rate a nonsense ISBN.
\item \textit{testGiveInvalidRating} test that only valid ratings can be given, that is ratings between one and five inclusive. The book-stores state should not change if an invalid rating is given.
\item \textit{testGiveValidRating} test that given valid ratings of books in the store the rating of the book is aggregated. To perform this test a mutable book is added to the book-store, this book is then rated and checked that the rating of this book is updated expected. Finally we remove the book from the book-store and check that nothing else has changed.
\end{itemize}
We have implemented the following test to demonstrate correctness of the implemented \textit{rateBooks} method:
\begin{itemize}
\item \textit{testTopRatedNegativeNumBooks} test that we cannot ask for less than one top rated books by asking for  a negative number of books. The \textit{getTopRatedBooks} method is a read only method and hence should never change the state of the book-store. 
\item \textit{testKTopRated} test that we the k highest rated books, where \(k\epsilon{1,2,3}\). The books are rated such that the test also checks that the average book ratings are used and not the aggregate, or total, book rating.
\item \textit{testAllTopRatedBook} test that if we query for more than the number of books in the book-store all books in the book-store are returned.
\end{itemize}
We have implemented the following test to demonstrate correctness of the implemented \textit{getBooksInDemand} method:
\begin{itemize}
\item \textit{tesNoBooksInDemand} test that if there are no books in demand none are returned. The \textit{getBooksInDemand} method is read only and should therefore not change the state of the bookstore. We check the consistency of the book-store by checking that a before and after snapshot of the book-store are the same.
\item \textit{testNoBooksInBookStore} test that no books are in demand if there are no books in the store. Like \textit{testNoBooksInDemand} this state of the book-store service should be unchanged before and after calling \textit{getBooksInDemand}.
\item \textit{testNoBooksInDemand} test that books in demand are returned. This method first generates a sales miss on a book in the store and then check that this book is added to the list of books in demand. We take the before snapshot after generating the salesmiss, as we are not testing the sematics of \textit{buyBooks}, and then compare to the after state, which as with the other should remain unchanged.
\end{itemize}
All test behave as expected in the implementation.
\paragraph{The implementation} presents a solution for implementing three methods in the business logic, \textit{rateBooks}, \textit{getTopRatedBooks} and \textit{getBooksInDemand}, and their corresponding stubs to perform RPC. The semantics of the business logic methods follow the specifications of the assignment. To insure all or nothing semantics all input variables are validated against the specifications of the book-store system before the request is committed for processing. As input arguments is the only user interaction, the semantics and syntax of these are the only unknown variables that needs verification at runtime for all or nothing semantics. To perform the RPC we extend the \textit{handler} method in the \texttt{BookStoreHTTPMessageHandler} class to handle the two keywords: RATEBOOKS and GETTOPRATEDBOOKS and implements stubs for to serialize parameters send by the client in the \texttt{BookStoreHTTPProxy} class. The handler will make the appropriate call to the business logic and serialize the output so that it can be sent over the network. For the third method \textit{getBooksInDemand} the server side implementation of for RPC follow that same pattern of introducing a new messagetag calling business logic and serilizing the output. The client side stub is implemented in the \texttt{StockManagerHTTPProxy} class, as appropriate.
\subsection{Modularity and Isolation}
The architecture achieves strong modularity through deploying a client server model and implementing RPCs to hide communication. The clients in the architecture communicate though proxies to the a HTTP server power by Jetty, this server deploys thread safe method calls, using the \textbf{synchronized} key word, to our business logic. The client server module achieves modularity in that the client is only exposed to the API of our book-store. RPCs also provide modularity in that they allow a entire network to lie behind the exposed book-store API.\\
The book-store client and stock-manager client are fully isolated, there is no direct communication between the two. The two clients both affect the state of the book-store and so the may affect the response to request to the book-store, but one crashing will not make the book-store service available to the other. Furthermore clients do not need to consider the effects of concurrent execution when constructing their request. The book-store service is also isolated from the clients if they are deployed on separate machines. That is if the book-store service becomes unavailable or crashes will render requests moot, but will no make the clients unavailable.\\
We cannot provide the isolation and modularity guarantees when performing local testing as the JVM is a single virtual machine. That is a fatal error in a client or the server will bring down the entire running JVM instance.
\subsection{Naming service}

\subsection{Exactly-once RPC semantics}

\subsection{Usage of proxy servers}

\subsection{Bottlenecks}

\subsection{Consequences of a server-side crash}

\end{document}
